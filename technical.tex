\section{Design and Implementation}
\label{sec:design-impl}

\texttt{z3.rkt} is currently implemented as a few hundred lines of Racket code
that interface with the Z3 engine via the provided library. Since the system is
still a work in progress, some of these details might change in the future.

\subsubsection{The Z3 wrapper.} We use Racket's foreign interface \cite{racket/foreign}
to map the Z3 library's C functions into Racket. The programmer interface
communicates with Z3 by calling the Racket functions defined by the
wrapper. While it is possible to use the Z3 wrapper directly, we highly
recommend using the programmer interface instead.

\subsubsection{Built-in functions.} Z3 comes with a number of built-in functions that
operate on booleans, numbers, and more complex values. We expose these
functions directly but add a \texttt{/s} suffix to their usual names in the
SMT-LIB standard, because most SMT-LIB names are already defined as functions
by Racket and we want to avoid colliding with them.

\subsubsection{The core commands.} This is a small set of Racket macros and functions
layered on top of the Z3 wrapper. As noted in Section~\ref{sec:motiv}, the aim
here is to hide the complexities of the C wrapper and stay as close to SMT-LIB
version 2 commands \cite{smtlib2:10} as possible. We prefix commands with
\texttt{smt:} to avoid collisions with Racket functions.

\subsection{Derived Abstractions}
\label{sec:derived}

Since the full power of Racket is available to us, we can define abstractions
that allow users to simplify their code. For example, SMT-LIB allows users to
define macros via the \texttt {define-fun} command.

\begin{verbatim}
(define-fun max ((a Int) (b Int)) Int
  (ite (> a b) a b)) ; ite is short for if-then-else
...
(assert (= (max 4 7) 7))
\end{verbatim}

Our first attempt to implement this facility for \texttt{z3.rkt} was to define
a Racket function to do the same thing:

\begin{verbatim}
(define (smt-max a b)
  (ite/s (>/s a b) a b))
...
(smt:assert (=/s (smt-max 4 7) 7))
\end{verbatim}

This works for smaller macros like \texttt{max}, but in our experience this
sort of na\"{i}ve substitution can result in final expressions for deeply
nested functions becoming too large for Z3 to handle. Consider a function
\texttt{(f a b)} that uses \texttt{(smt-max a b)} $m$ times. This expression
will be repeated $m$ times in the expression for \texttt{f}. Now consider a
function \texttt{g} that uses the value of \texttt{(f 0 1)} $n$ times. The
expression for \texttt{g} will contain that of \texttt{(f 0 1)} $n$ times, and
consequently that of \texttt{(smt-max 0 1)} $m n$ times. As the nesting
increases, we see an exponential blowup in the size of the final
expression\footnote{We could merge common parts of expressions to reduce the
number of AST nodes generated. In our experiments, this proved to be quite
effective, yet still significantly slower than the solution we finally
adopted.}.

We note, however, that any macro can also be written as a universally
quantified formula. For example, \texttt{max} can be rewritten in the
following way.

\begin{verbatim}
(declare-fun max (Int Int) Int)
(assert (forall ((a Int) (b Int))
                (= (max a b)
                   (ite (> a b) a b))))
\end{verbatim}

We finally solved the problem by providing a Racket macro,
\texttt{smt:define-fun}, that has the same syntax as the SMT-LIB command
and that outputs the equivalent universally quantified formula.

\subsection{Porting Existing SMT-LIB Code}
\label{sec:porting-smt-lib}

One of our explicit goals is to enable existing SMT-LIB version 2 code to be
ported with a small number of systematic changes. Table~\ref{table:smt-porting}
lists the minimal set of changes that needs to be made to port
existing SMT-LIB code to \texttt{z3.rkt}. We expect many SMT-LIB programs
to become shorter as authors use Racket features wherever appropriate.

\begin{table}[hbt]
\caption{Differences between SMT-LIB and \texttt{z3.rkt}}
\label{table:smt-porting}
\begin{center}
\begin{tabularx}{0.91\textwidth}{lX}
\hline\noalign{\smallskip}
SMT-LIB code & \texttt{z3.rkt} code \\
\noalign{\smallskip}
\hline
\noalign{\smallskip}
Options: \texttt{(set-option :foo true)} & Keyword arguments: \newline \texttt{(smt:new-context \#:foo \#t)} \\

Logics: \texttt{(set-logic QF\_UF)} & The \texttt{\#:logic} keyword: \newline \texttt{(smt:new-context \#:logic "QF\_UF")} \\

Commands: \texttt{declare-fun}, \texttt{assert}, \ldots & Prefixed with \texttt{smt:} \\

Functions: \texttt{and}, \texttt{or}, \texttt{+}, \texttt{distinct} \ldots & Suffixed with \texttt{/s} \\

Boolean literals: \texttt{true} and \texttt{false} & \texttt{\#t} and \texttt{\#f} \\

% Bit-vector literals: \texttt{\#b101}, \texttt{\#x4d56} & As strings: \texttt{"\#b101"}, \texttt{"\#x4d56"} \\
\hline
\end{tabularx}
\end{center}
\end{table}
